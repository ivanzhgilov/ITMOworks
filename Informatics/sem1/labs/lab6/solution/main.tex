\documentclass[12pt]{article}
\usepackage[utf8]{inputenc}
\usepackage[T2A]{fontenc}
\usepackage[russian]{babel}
\usepackage{amsmath}
\usepackage{graphicx}
\usepackage{caption}
\usepackage{multicol}
\usepackage{float}
\usepackage{fancyhdr}
\usepackage[margin=1.5cm]{geometry}
\usepackage{xcolor}

\graphicspath{ {images/} }

\pagestyle{fancy}
\renewcommand{\headrulewidth}{0pt}
\fancyhead{}
\fancyfoot{}
\fancyfoot[L]{\textbf{\thepage}}
\setlength{\columnsep}{50pt}

\setcounter{page}{8}
\setcounter{figure}{3}


\begin{document}

\begin{multicols}{2}

\noindent Самой простой из всех систем записи целых чисел является \emph{двоичная} система счисления, в которой все числа представляются суммами степеней числа 2. Если использовать эту систему счисления, то наши таблицы I и II примут совсем другой вид (числа в двоичной системе счисления мы будем записывать красными цифрами):

\end{multicols}

\begin{center}
Таблица Ia

\begin{tabular}{|c|*{15}{c|}}
\hline

№№ & 1 & 2 & 3 & 4 & 5 & 6 & 7 & 8 & 9 & 10 & 11 & 12 & 13 & 14 & 15 \\
\hline
a & \color{red}0 & \color{red}0 & \color{red}1 & \color{red}0 & \color{red}0 & \color{red}1 & \color{red}0 & \color{red}0 & \color{red}0 & \color{red}0 & \color{red}0 & \color{red}0 & \color{red}0 & \color{red}0 & \color{red}0 \\
\hline
b & \color{red}1 & \color{red}10 & \color{red}10 & \color{red}11 & \color{red}100 & \color{red}100 & \color{red}101 & \color{red}100 & \color{red}110 & \color{red}100 & \color{red}101 & \color{red}110 & \color{red}111 & \color{red}1000 & \color{red}1000 \\
\hline
c & \color{red}1 & \color{red}10 & \color{red}11 & \color{red}11 & \color{red}100 & \color{red}101 & \color{red}101 & \color{red}110 & \color{red}110 & \color{red}111 & \color{red}111 & \color{red}111 & \color{red}111 & \color{red}1000 & \color{red}1001 \\
\hline
\end{tabular}
\end{center}

\begin{center}
Таблица IIa

\begin{tabular}{|c|*{8}{c|}}
\hline
№№ & 1 & 2 & 3 & 4 & 5 & 6 & 7 & 8 \\
\hline
a & \color{red}1 & \color{red}11 & \color{red}100 & \color{red}110 & \color{red}1000 & \color{red}1001 & \color{red}1011 & \color{red}1000 \\
\hline
b & \color{red}10 & \color{red}101 & \color{red}111 & \color{red}1010 & \color{red}1101 & \color{red}1111 & \color{red}1010 & \color{red}1010 \\
\hline
\end{tabular}
\end{center}

\begin{center}

\begin{tabular}{|c|*{6}{c|}}
\hline
9 & 10 & 11 & 12 & 13 & 14 & 15 \\
\hline
\color{red}1110 & \color{red}10000 & \color{red}10001 & \color{red}10011 & \color{red}10101 & \color{red}10110 & \color{red}11000 \\
\hline
\color{red}10111 & \color{red}11010 & \color{red}11100 & \color{red}11111 & \color{red}100010 & \color{red}100100 & \color{red}100111 \\
\hline
\end{tabular}
\end{center}

\begin{multicols}{2}

Внимательное изучение таблицы Ia позволяет усмотреть в ней вполне определенную закономерность: да, конечно, \emph{проигрышные позиции (a, b, c) в игре «ним» -- это те позиции, где сумма цифр каждого разряда в двоичной записи чисел a, b, c четная} (то есть равна 0 или 2)! Однако таблица IIa оказывается более коварной: она ничем не лучше таблицы II, и никакой определенной закономерности составления проигрышных пар (a, b) из нее усмотреть нельзя.

Но почему именно двоичная система счисления должна дать нам ключ к отысканию системы беспроигрышной игры в цзиньшицзы? Правда, мы видели, что эта система помогает как будто при разработке теории игры ним: но ведь цзиньшицзы -- это совсем другая игра, а ключ, открывающий одну дверь, совсем не обязан подходить к другой. Нельзя ли переписать таблицу II по-иному: может быть, другая система записи образующих эту систему чисел прольет больше света на ее строение?

Внимательное изучение таблицы II позволяет подметить, что в ней часто фигурируют числа Фибоначчи (см., например, 1-й, 2-й, 5-й и 13-й столбцы таблицы *). А это в свою очередь может навести нас на мысль о целесообразности попытки записать таблицу II в «фибоначчиевой системе счисления», в которой все числа представляются суммами чисел последовательности Фибоначчи (числа в этой системе счисления мы будем записывать синими цифрами). Например, число 100 в фибоначчиевой

\noindent\rule{0.2\columnwidth}{0.5pt}

)* Числа Фибоначчи — последовательность чисел, начинающая с 1 и 2 и далее строящаяся по следующему закону: \emph{каждое число последовательности Фибоначчи равно сумме двух предыдущих.} Первые ее члены таковы: 1, 2, 3, 5, 8, 13, 21, 34, 55, 89, 144, 233, 377, ... Числам Фибоначчи посвящена хорошая брошюра Н. Н. Воробьева «Числа Фибоначчи» (М., «Наука», 3-е изд., 1969).

\end{multicols}

\newpage
\setcounter{page}{18}

\begin{multicols}{2}

\begin{figure}[H]
    \centering
    \includegraphics[width=0.8\linewidth]{graph1}
    \caption{}
    \label{fig:2}
\end{figure}

\noindent науки, но и для техники. В знаменитых квантовых генераторах -- лазерах и мазерах -- специально создают отрицательную температуру в том именно смысле, о котором вы говорили, а затем быстро «высвечивают» скопленную энергию.

\noindent\textbf{Читатель}

Разве температура определяет распределение частиц по энергетическим уровням только в квантовой механике?

\noindent\textbf{Автор}

Конечно нет. Это верно и в классической механике. Например, распределение по энергиям молекул «классического» идеального газа такое, как показано на рисунке 4. Оно определяется формулой

\[
\frac{n}{n_0} = e^{-\frac{E}{kT}}
\]

Здесь \( n_0 \) — число частиц в том состоянии, с которого мы начинаем отсчет энергии частицы \( E \), \( n \) — это число частиц с энергией \( E \); \( e = 2,718 \) — основание натуральных логарифмов, \( k \) — постоянная Больцмана, \( T \) — температура газа.

\noindent\textbf{Читатель}

Если преобразовать эту формулу, прологарифмировав обе части равенства, то она примет вид

\[
\ln \frac{n}{n_0} = -\frac{E}{kT}
\]

\begin{figure}[H]
    \centering
    \includegraphics[width=0.8\linewidth]{graph2} 
    \caption{}
    \label{fig:5}
\end{figure}

\noindentОтсюда

\[
T = -\frac{E}{k (\ln n - \ln n_0)}
\]

Если к газу подводить энергию, то мы будем повышать его температуру, переводя все больше частиц с нижних энергетических уровней на верхние. При этом мы можем добиться того, что \( n \) станет больше \( n_0 \) и \(\ln n\) больше \(\ln n_0\). Это будет означать, что температура газа отрицательна! Мне кажется, мы пришли к абсурду. Говорить об отрицательной температуре идеального газа, по-моему, бессмысленно.

\noindent\textbf{Автор}

Если мы будем рассматривать только несколько энергетических уровней, то мы можем сделать так, что на верхних из них будет больше молекул газа, чем на нижних. Однако нельзя добиться того, чтобы распределение молекул по энергиям было таким, как показано на рисунке 5: зависимость числа частиц от энергии экспоненциальная и на достаточно высоких уровнях находится больше частиц, чем на нижних.

Все дело в том, что наш вывод о возможности отрицательных температур справедлив не для любой системы частиц, а только для такой, у

\end{multicols}

\end{document}